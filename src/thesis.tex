%
% Omigost Project
%
% MIT License
% Copyright 2018
%
% Piotr Styczyński     (Univeristy of Warsaw)
% Michał Balcerzak     (Univeristy of Warsaw)
% Michał Ołtarzewski   (Univeristy of Warsaw)
% Gor Safaryn          (Univeristy of Warsaw)
%
% Permission is hereby granted, free of charge, to any person obtaining a copy of this software and associated documentation files (the "Software"),
% to deal in the Software without restriction, including without limitation the rights to use, copy, modify, merge, publish, distribute, sublicense,
% and/or sell copies of the Software, and to permit persons to whom the Software is furnished to do so, subject to the following conditions:
%
% The above copyright notice and this permission notice shall be included in all copies or substantial portions of the Software.
%
% THE SOFTWARE IS PROVIDED "AS IS", WITHOUT WARRANTY OF ANY KIND, EXPRESS OR IMPLIED, INCLUDING BUT NOT LIMITED TO THE WARRANTIES OF MERCHANTABILITY,
% FITNESS FOR A PARTICULAR PURPOSE AND NONINFRINGEMENT. IN NO EVENT SHALL THE AUTHORS OR COPYRIGHT HOLDERS BE LIABLE FOR ANY CLAIM,
% DAMAGES OR OTHER LIABILITY, WHETHER IN AN ACTION OF CONTRACT, TORT OR OTHERWISE, ARISING FROM,
% OUT OF OR IN CONNECTION WITH THE SOFTWARE OR THE USE OR OTHER DEALINGS IN THE SOFTWARE.
%
%
%

% Add option for language formatting en/pl
\documentclass[licencjacka,en]{thesisclass}
\usepackage{standalone}
\usepackage{import}
\usepackage{graphicx}
\usepackage{grffile}
\usepackage{svg}
\usepackage{calc}

% TODO
% Authors of the thesis:
\autor{Piotr Styczyński}{000000}
\autori{Michał Balcerzak}{000000}
\autorii{Michał Ołtarzewski}{000000}
\autoriii{Gor Safaryn}{000000}

\title{AWS Cost Optimization Tool - Omigost}
\titlepl{Narzędzie Optymalizacji Kosztów AWS Omigost}

%\tytulang{Implementation of services for reliable AWS services cost tracking, budget management and heuristic optimalizations}

% The main degree
\kierunek{Computer Science}

% The thesis supervisor
\opiekun{dr Janina Zofia Mincer-Daszkiewicz\\
  Instytut Informatyki\\
  }

% Date in format <month> <year>
\date{May 2018}

% Doctrine of classification as it states the Socrates-Erasmus:
% TODO
\dziedzina{ 
%11.0 Matematyka, Informatyka:\\ 
%11.1 Matematyka\\ 
% 11.2 Statystyka\\ 
11.3 Informatyka\\ 
%11.4 Sztuczna inteligencja\\ 
%11.5 Nauki aktuarialne\\
%11.9 Inne nauki matematyczne i informatyczne
}

% TODO
% Subject classification due to ACM
\klasyfikacja{D. Software\\
  D.127. Blabalgorithms\\
  D.127.6. Numerical blabalysis}

% Keywords list
\keywords{system, aws, cost, optimization, heuristic, cost management, koszt, aws, optymalizacja, heurystyka kosztowa}

% TODO
% Place for custom definitions and environments
\newtheorem{defi}{Definicja}[section]

% End of definitions

\begin{document}
\maketitle

% Brief for the first page (short abstract)
\begin{abstract}
  \input{src/abstract.tex}
\end{abstract}

\tableofcontents
%\listoffigures
%\listoftables

\chapter{Introduction}
\addcontentsline{toc}{chapter}{Introduction}
\subimport{./}{./src/introduction.tex}

\chapter{Problem statement}

\includegraphics[width=\textwidth*\real{0.4}]{imgs/sombrerro.png}

\section{Motivation}

\subsection{What is cloud computing}

Cloud computing became one of the most important paradigm shifts in the area of real world software engineering.
It has reshaped the whole process of how applications are developed, deployed and reduced the amount of upfront
investment required to start an internet business. While commercial cloud computing services were first offered
in 2006 by Amazon Inc, the original idea and preliminary implementation traces back to Multics OS developed by MIT,
GE and Bell Labs. However the idea of time-sharing systems that was the ancestor of further cloud
concept was widespread in 60ies (\cite{bib-cloud-markus}).

\subsection{The term itself}

The term “Cloud computing” can refer to every layer of application stack:
hardware, hosting platform, software and even to a single function.
Cloud Computing refers to both the applications delivered as services over the Internet and
the hardware and systems software in the data centers that provide those services.
The services themselves have long been referred to as Software as a Service (SaaS) (\cite{bib-cloud-armbrust}).
Namely, it is a shared pool of computer resources such as computing capacity,
transient and persistent memory, which can be acquired or released on demand.
The undisputed power of cloud computing constitutes in its elasticity and granularity:
i.e. it allows users to ask for hundreds of computers for only 5 minute usage which
are shipped during several minutes.
Such services are usually offered over remote network connection and users are billed
for the portion of the resources they have used.
Depending on the cloud infrastructure type the payment models can be different,
but the common spendings are associated with data storage, data transfer, and computing timeshare
\cite{bib-cloud-laatikainen}.

\subsection{Cloud resources}

Cloud computing has introduced 3 new aspects which are radically different from a traditional
computing paradigm in terms of hardware infrastructure. 
The availability of practically infinite on demand computing resources,
which allows users to deploy applications without any kind of resource planning.  
The elimination of massive initial financial commitment, therefore allowing small companies to increase
their hardware usage proportional to their needs.
The ability to granularly allocate computing resources on any kind of timeframe
(from minutes up to days etc.), as well as the ability to release them as there is no more need.

Depending on the service offered by the provider we differentiate 3 main models of cloud computing:
Infrastructure as a service (IAAS),
Platform as a service (PAAS) and
Software as a service(SAAS).

\subsection{IAAS - Infrastructure as a service}

In such form of service, the provider allocates an instance of virtual machine for the client 
and ensures that minimal building blocks for the IT infrastructure are present:
network, storage, computing capacity, load balancer, VLAN etc.
Usually providers of such services run pools of hypervisors such as Xen, VMware, QEMU etc which host and manage
those virtual machines. From the user perspective it usually looks like a command line interface
through which user has full control over the allocated virtual machine. Some of the well known
services are Amazon EC2, Google Compute Engine, Microsoft Azure IAAS.

\subsection{PAAS - Platform as a service}

Platform as a service alleviates the need for the developers to manage the operating system and
provides programming language specific execution environment as well as the underlying structure
(hardware, network, storage etc). This adds another layer of convenience over IAAS making the deployment
and the development of applications much more fluent process. While users benefit from the automatic software maintenance,
OS security patches etc, they also lose full control over the virtual machine instances. AWS Elastic Beanstalk, Heroku
and Google App Engine are some of the most popular PAAS services.

\subsection{SAAS - Software as a service}

This form of service takes control of every layer of application and the user has nothing to do with
the underlying infrastructure. This is by all means one of the most popular form of cloud computing as
it might not require any kind of technical background to be used. Examples of SAAS services are Google
Cloud Vision, Google Docs, Microsoft Office 365. 

\begin{thebibliography}{99}
\addcontentsline{toc}{chapter}{Bibliography}

\bibitem[Cloud01]{bib-cloud-armbrust}
Michael Armbrust Armando Fox Rean Griffith Anthony D. Joseph Randy H. Katz et al.
\textit{Above the Clouds: A Berkeley View of Cloud Computing} 
Technical Report No. UCB/EECS-2009-28

\bibitem[AWSDocs]{bib-aws-docs}
Amazon
\textit{Amazon AWS documentation} 
    
\bibitem[Cloud02]{bib-cloud-laatikainen}
Laatikainen, G., Ojala, A., Mazhelis, O. (2013)
\textit{Cloud Services Pricing Models}

\bibitem[Cloud03]{bib-cloud-markus}
MARKUS BÖHM, STEFANIE LEIMEISTER, CHRISTOPH RIEDL, HELMUT KRCMAR
\textit{Cloud Computing and Computing Evolution}
Technische Universität München (TUM), Germany

\end{thebibliography}

\end{document}


%%% Local Variables:
%%% mode: latex
%%% TeX-master: t
%%% coding: latin-2
%%% End:
