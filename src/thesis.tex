%
% Omigost Project
%
% MIT License
% Copyright 2018
%
% Piotr Styczyński     (Univeristy of Warsaw)
% Michał Balcerzak     (Univeristy of Warsaw)
% Michał Ołtarzewski   (Univeristy of Warsaw)
% Gor Safaryn          (Univeristy of Warsaw)
%
% Permission is hereby granted, free of charge, to any person obtaining a copy of this software and associated documentation files (the "Software"),
% to deal in the Software without restriction, including without limitation the rights to use, copy, modify, merge, publish, distribute, sublicense,
% and/or sell copies of the Software, and to permit persons to whom the Software is furnished to do so, subject to the following conditions:
%
% The above copyright notice and this permission notice shall be included in all copies or substantial portions of the Software.
%
% THE SOFTWARE IS PROVIDED "AS IS", WITHOUT WARRANTY OF ANY KIND, EXPRESS OR IMPLIED, INCLUDING BUT NOT LIMITED TO THE WARRANTIES OF MERCHANTABILITY,
% FITNESS FOR A PARTICULAR PURPOSE AND NONINFRINGEMENT. IN NO EVENT SHALL THE AUTHORS OR COPYRIGHT HOLDERS BE LIABLE FOR ANY CLAIM,
% DAMAGES OR OTHER LIABILITY, WHETHER IN AN ACTION OF CONTRACT, TORT OR OTHERWISE, ARISING FROM,
% OUT OF OR IN CONNECTION WITH THE SOFTWARE OR THE USE OR OTHER DEALINGS IN THE SOFTWARE.
%
%
%

% Add option for language formatting en/pl
\documentclass[licencjacka,en]{thesisclass}
\usepackage{standalone}
\usepackage{import}
\usepackage{graphicx}
\usepackage{grffile}
\usepackage{svg}
\usepackage{calc}
\usepackage[backref=true,                % 
            hyperref=true,               % 
            firstinits=true,             %
            indexing=true,               %
            url=false,                   % 
            style=alphabetic,            %  style=debug, alphabetic
            backend=biber,               % 
            doi=false,
            texencoding=utf8,
            bibencoding=utf8]{biblatex} 

\addbibresource{src/thesis.bib}    
\usepackage{biblatex}
\addbibresource{src/thesis.bib} 

% TODO
% Authors of the thesis:
\autor{Piotr Styczyński}{386038}
\autori{Michał Balcerzak}{385130}
\autorii{Michał Ołtarzewski}{382783}
\autoriii{Gor Safaryan}{381501}

\title{AWS Cost Optimization Tool - Omigost}
\titlepl{Narzędzie Optymalizacji Kosztów AWS Omigost}

%\tytulang{Implementation of services for reliable AWS services cost tracking, budget management and heuristic optimalizations}

% The main degree
\kierunek{Computer Science}

% The thesis supervisor
\opiekun{dr Janina Zofia Mincer-Daszkiewicz\\
  Instytut Informatyki\\
}

% Date in format <month> <year>
\date{May 2018}

% Doctrine of classification as it states the Socrates-Erasmus:
\dziedzina{  
11.3 Informatyka\\
}

% TODO
% Subject classification due to ACM
\klasyfikacja{D. Software}

% Keywords list
\keywords{system, aws, cost, optimization, heuristic, cost management, koszt, aws, optymalizacja, heurystyka kosztowa}

% TODO
% Place for custom definitions and environments
\newtheorem{defi}{Definicja}[section]

% End of definitions

\usepackage{lmodern}

\begin{document}
\maketitle

% Brief for the first page (short abstract)
\begin{abstract}
  \input{src/abstract.tex}
\end{abstract}

\chapter*{Contents}

\begin{enumerate}
    \item Introduction
        \begin{enumerate}
            \item [1.1] Overview
            \item [1.2] Structure of thesis 
        \end{enumerate}
    \item Problem statement
        \begin{enumerate}
            \item [2.1] Motivation
            \item [2.2] Overview of exisiting solutions
            \item [2.3] Our solution
        \end{enumerate}
    \item Tool for AWS cost optimisation
        \begin{enumerate}
            \item [3.1] Technologies stack
            \item [3.2] Overall Architecture
            \item [3.3] Data models
            \item [3.4] Views and design
            \item [3.5] API
            \item [3.6] Communication integration
            \item [3.7] Configuration
        \end{enumerate}
    \item Summary
    \item [A] Deployment and integration guide
\end{enumerate}

\chapter{Introduction}

    % \includegraphics[width=\textwidth*\real{0.4}]{imgs/logo.png}

    \section{Overview}

        \subsection{What is cloud computing}

        Cloud computing became one of the most important paradigm shifts in the area of real world software engineering.
        It has reshaped the whole process of how applications are developed, deployed and reduced the amount of upfront
        investment required to start an internet business. While commercial cloud computing services were first offered
        in 2006 by Amazon Inc, the original idea and preliminary implementation traces back to Multics OS developed by MIT,
        GE and Bell Labs. However the idea of time-sharing systems that was the ancestor of further cloud
        concept was widespread in 60ies~\cite{BMarkus}.

        \subsection{The term itself}

        The term “Cloud computing” can refer to every layer of application stack:
        hardware, hosting platform, software and even to a single function.
        Cloud Computing refers to both the applications delivered as services over the Internet and
        the hardware and systems software in the data centers that provide those services.
        The services themselves have long been referred to as Software as a Service (SaaS)~\cite{MArmbrust}.
        Namely, it is a shared pool of computer resources such as computing capacity,
        transient and persistent memory, which can be acquired or released on demand.
        The undisputed power of cloud computing constitutes in its elasticity and granularity:
        i.e. it allows users to ask for hundreds of computers for only 5 minute usage which
        are shipped during several minutes.
        Such services are usually offered over remote network connection and users are billed
        for the portion of the resources they have used.
        Depending on the cloud infrastructure type the payment models can be different~\cite{GLaatikainen},
        but the common spendings are associated with data storage, data transfer, and computing timeshare.

        \subsection{Cloud resources}

        Cloud computing has introduced 3 new aspects which are radically different from a traditional
        computing paradigm in terms of hardware infrastructure. 
        The availability of practically infinite on demand computing resources,
        which allows users to deploy applications without any kind of resource planning.  
        The elimination of massive initial financial commitment, therefore allowing small companies to increase
        their hardware usage proportional to their needs.
        The ability to granularly allocate computing resources on any kind of timeframe
        (from minutes up to days etc.), as well as the ability to release them as there is no more need.

        Depending on the service offered by the provider we differentiate 3 main models of cloud computing:

        \begin{enumerate}
            \item Infrastructure as a service (IAAS)
            \item Platform as a service (PAAS)
            \item Software as a service(SAAS)
        \end{enumerate}

        \subsection{IAAS - Infrastructure as a service}

        In such form of service, the provider allocates an instance of virtual machine for the client 
        and ensures that minimal building blocks for the IT infrastructure are present:
        network, storage, computing capacity, load balancer, VLAN etc.
        Usually providers of such services run pools of hypervisors such as Xen, VMware, QEMU etc which host and manage
        those virtual machines. From the user perspective it usually looks like a command line interface
        through which user has full control over the allocated virtual machine. Some of the well known
        services are Amazon EC2, Google Compute Engine, Microsoft Azure IAAS.

        \subsection{PAAS - Platform as a service}

        Platform as a service alleviates the need for the developers to manage the operating system and
        provides programming language specific execution environment as well as the underlying structure
        (hardware, network, storage etc). This adds another layer of convenience over IAAS making the deployment
        and the development of applications much more fluent process. While users benefit from the automatic software maintenance,
        OS security patches etc, they also lose full control over the virtual machine instances. AWS Elastic Beanstalk, Heroku
        and Google App Engine are some of the most popular PAAS services.

        \subsection{SAAS - Software as a service}

        This form of service takes control of every layer of application and the user has nothing to do with
        the underlying infrastructure. This is by all means one of the most popular form of cloud computing as
        it might not require any kind of technical background to be used. Examples of SAAS services are Google
        Cloud Vision, Google Docs, Microsoft Office 365. 

\chapter{Problem statement}

    \section{Motivation}

        As there are plenty of various billing models for cloud services~\cite{GLaatikainen}
        the effective management of them became a tough problem.
        The ease of resource allocation led to situation when tracking tiniest details of billings is an unaffordable challenge.

        Many of bussiness are utilizing dedicated teams for cost management or use specialized tools i.e.
        Cloudability~\cite{Cloudability}, Apptio~\cite{Aptio}, Snow Software~\cite{SnowSoftware} and many more.

        The tooling that exists is targetting wideworld-scale comapnies that are able to require expensive licenses and hire cost-optimization teams.
        The software as it is in case of Cloudability is complex for average user and do not provide easy way to incorporate custom business flows into the tool.
        Amazon as one on the leading cloud providers offers tools for exploration of expenses including public APIs~\cite{AWSCostManagement},
        but the tools are rather simple and does not satisfy all the needs of potential clients.

        The common case that is unresolved is the distribution of RnD and development resources.
        We observed there exist no tool that would support request for resources of individual worker with regards to custom management propagations as specified by client bussiness model.
        Cloudability~\cite{CloudabilityAlerts} offers simple alerts, but they lack Slack support and beforementioned propagation abilities.

        As there exist an obvious gap in the market, Omigost tries to provide these unique features to the client:

        \begin{enumerate}
            \item Easy cloud management without expensive licenses or complex knowledge
            \item Ability to provide intuitive interface for individual workers to request resources
            \item Painless integration with existing business flows
        \end{enumerate}

    \section{Overview of exisiting solutions}
    
        Overview of available solutions for optimising AWS costs
        Businesses that choose to rely upon cloud services often reach a point where resources they’re using up become less and less manageable. They are 
        
        \begin{enumerate}
            \item Both AWS and third party
            \item Various niches (budget alerting, instance alerting management, cost analytics)
        \end{enumerate}

        Tools available on market:
        
        \begin{enumerate}
            \item AWS Budgets
        \end{enumerate}
        
        \subsection{AWS Budgets}

        AWS Budgets is a part of Amazon Web Services that allows to set limits of a certain types
        that apply to a chosen period of time. When a limit is either exceeded,
        close to be exceeded or is forecasted to exceed the configured threshold before the end of that period,
        the administrator of that account is notified by email. Types of resources one can put this kind of a budget on include:
        
        \begin{enumerate}
            \item Money spent in total or on a certain type of machines
            \item Utilisation of selected services
            \item Utilisation or coverage of reserved instances
        \end{enumerate}

        % TODO: Remove that?
        % Stax.io
        % AWS Cost Explorer and Management
        % https://aws.amazon.com/aws-cost-management/reserved-instance-reporting/

    
\begin{thebibliography}{99}
    \addcontentsline{toc}{chapter}{Bibliography}

    \bibitem[MArmbrust]{MArmbrust}
    Michael Armbrust Armando Fox Rean Griffith Anthony D. Joseph Randy H. Katz et al.
    \textit{Above the Clouds: A Berkeley View of Cloud Computing} 
    Technical Report No. UCB/EECS-2009-28

    \bibitem[AWSDocs]{AWSDocs}
    Amazon
    \textit{Amazon AWS documentation} 
        
    \bibitem[GLaatikainen]{GLaatikainen}
    Laatikainen, G., Ojala, A., Mazhelis, O. (2013)
    \textit{Cloud Services Pricing Models}

    \bibitem[BMarkus]{BMarkus}
    MARKUS BÖHM, STEFANIE LEIMEISTER, CHRISTOPH RIEDL, HELMUT KRCMAR
    \textit{Cloud Computing and Computing Evolution}
    Technische Universität München (TUM), Germany

    \bibitem[Cloudability]{Cloudability}
    \textit{https://www.cloudability.com/product/transform/}

    \bibitem[CloudabilityAlerts]{CloudabilityAlerts}
    \textit{https://blog.cloudability.com/creating-budget-alerts-by-tag-with-cloudability/}

    \bibitem[Aptio]{Aptio}
    \textit{https://www.apptio.com/}

    \bibitem[SnowSoftware]{SnowSoftware}
    \textit{https://go.snowsoftware.com/}

    \bibitem[AWSCostManagement]{AWSCostManagement}
    \textit{https://docs.aws.amazon.com/aws-cost-management/latest/APIReference/Welcome.html}

\end{thebibliography}

\end{document}


%%% Local Variables:
%%% mode: latex
%%% TeX-master: t
%%% coding: latin-2
%%% End:
