%
% Omigost Project
%
% MIT License
% Copyright 2018
%
% Piotr Styczyński     (University of Warsaw)
% Michał Balcerzak     (University of Warsaw)
% Michał Ołtarzewski   (University of Warsaw)
% Gor Safaryan         (University of Warsaw)
%
% Permission is hereby granted, free of charge, to any person obtaining a copy of this software and associated documentation files (the "Software"),
% to deal in the Software without restriction, including without limitation the rights to use, copy, modify, merge, publish, distribute, sublicense,
% and/or sell copies of the Software, and to permit persons to whom the Software is furnished to do so, subject to the following conditions:
%
% The above copyright notice and this permission notice shall be included in all copies or substantial portions of the Software.
%
% THE SOFTWARE IS PROVIDED "AS IS", WITHOUT WARRANTY OF ANY KIND, EXPRESS OR IMPLIED, INCLUDING BUT NOT LIMITED TO THE WARRANTIES OF MERCHANTABILITY,
% FITNESS FOR A PARTICULAR PURPOSE AND NONINFRINGEMENT. IN NO EVENT SHALL THE AUTHORS OR COPYRIGHT HOLDERS BE LIABLE FOR ANY CLAIM,
% DAMAGES OR OTHER LIABILITY, WHETHER IN AN ACTION OF CONTRACT, TORT OR OTHERWISE, ARISING FROM,
% OUT OF OR IN CONNECTION WITH THE SOFTWARE OR THE USE OR OTHER DEALINGS IN THE SOFTWARE.
%
%
%

% Add option for language formatting en/pl
\documentclass[licencjacka,en]{thesisclass}
\usepackage{standalone}
\usepackage{import}
\usepackage{graphicx}
\usepackage{grffile}
\usepackage{svg}
\usepackage{calc}
% json formatting
\usepackage{listings}
\usepackage{xcolor}
\colorlet{punct}{red!60!black}
\definecolor{background}{HTML}{EEEEEE}
\definecolor{delim}{RGB}{20,105,176}
\colorlet{numb}{magenta!60!black}
\lstdefinelanguage{json}{
    basicstyle=\normalfont\ttfamily,
    numberstyle=\scriptsize,
    stepnumber=1,
    numbersep=8pt,
    showstringspaces=false,
    breaklines=true,
    backgroundcolor=\color{white},
    literate=
     *{0}{{{\color{numb}0}}}{1}
      {1}{{{\color{numb}1}}}{1}
      {2}{{{\color{numb}2}}}{1}
      {3}{{{\color{numb}3}}}{1}
      {4}{{{\color{numb}4}}}{1}
      {5}{{{\color{numb}5}}}{1}
      {6}{{{\color{numb}6}}}{1}
      {7}{{{\color{numb}7}}}{1}
      {8}{{{\color{numb}8}}}{1}
      {9}{{{\color{numb}9}}}{1}
      {:}{{{\color{punct}{:}}}}{1}
      {,}{{{\color{punct}{,}}}}{1}
      {\{}{{{\color{delim}{\{}}}}{1}
      {\}}{{{\color{delim}{\}}}}}{1}
      {[}{{{\color{delim}{[}}}}{1}
      {]}{{{\color{delim}{]}}}}{1},
}



\usepackage[backref=true,                %
hyperref=true,               %
firstinits=true,             %
indexing=true,               %
url=false,                   %
style=alphabetic,            %  style=debug, alphabetic
backend=biber,               %
doi=false,
texencoding=utf8,
bibencoding=utf8]{biblatex}

\addbibresource{src/thesis.bib}
\usepackage{biblatex}
\addbibresource{src/thesis.bib}

% Authors of the thesis:
\autor{Piotr Styczyński}{386038}
\autori{Michał Balcerzak}{385130}
\autorii{Michał Ołtarzewski}{382783}
\autoriii{Gor Safaryan}{381501}

\title{AWS Cost Optimization Tool}
\titlepl{Narzędzie do Optymalizacji Kosztów AWS}

% The main degree
\kierunek{Computer Science}

% The thesis supervisor
\opiekun{dr Janina Mincer-Daszkiewicz\\
Instytut Informatyki\\
}

% Date in format <month> <year>
\date{February 2019}

% Doctrine of classification as it states the Socrates-Erasmus:
\dziedzina{
11.3 Informatics, Computer Science\\
}

% TODO
% Subject classification due to ACM
\klasyfikacja{D. Software}

% Keywords list
\keywords{AWS, Amazon Web Services, cloud computing, cost optimization, cost management}

% TODO
% Place for custom definitions and environments
\newtheorem{defi}{Definicja}[section]

% End of definitions

\usepackage{lmodern}

\begin{document}
    \maketitle

    % Brief for the first page (short abstract)
    \begin{abstract}
        \input{src/abstract.tex}
    \end{abstract}

    \chapter*{Contents}

    \begin{enumerate}
        \item Introduction
        \begin{enumerate}
            \item [1.1] Overview
            \item [1.2] Aim of the thesis
            \item [1.3] Structure of the thesis
            \item [1.4] Contribution of each author
        \end{enumerate}
        \item Problem statement
        \begin{enumerate}
            \item [2.1] Motivation
            \item [2.2] Overview of exisiting solutions
            \item [2.3] Our solution
        \end{enumerate}
        \item Project development
        \begin{enumerate}
            \item [3.1] Version control
            \item [3.2] Continuous integration
            \item [3.3] Communication
            \item [3.4] Workflow
        \end{enumerate}
        \item Tool for AWS cost optimization
        \begin{enumerate}
            \item [4.1] Use cases
            \item [4.2] Overall architecture
            \item [4.3] Technology stack
            \item [4.4] Data models
            \item [4.5] Views and design
            \item [4.6] Communication integration
            \item [4.7] Service termination architecture
        \end{enumerate}
        \item Summary
        \item [A] Deployment and integration guide
    \end{enumerate}

    \chapter{Introduction}

    % \includegraphics[width=\textwidth*\real{0.4}]{imgs/logo.png}

    \section{Overview}

    Cloud computing has become recently one of the most important paradigm shifts in the area of real world software engineering.
    It has reshaped the whole process of how applications are developed and reduced the amount of upfront
    investment required to start an internet business.
    While commercial cloud computing services were first offered
    in 2006 by Amazon Inc, the original idea and preliminary implementation traces back to Multics OS developed by MIT,
    GE and Bell Labs.
    However the idea of time-sharing systems that was the ancestor of further cloud
    concept was widespread in 60ies~\cite{Markus}.

    The term “Cloud computing” can refer to every layer of application stack:
    hardware, hosting platform, software and even to a single function.
    Cloud computing refers to both the applications delivered as services over the Internet and
    the hardware and systems software in the data centers that provide these services.
    The services themselves have long been referred to as Software as a Service (SaaS)~\cite{Armbrust}.
    Generally speaking cloud can be perceived as a shared pool of computer resources such as computing capacity,
    transient and persistent memory, which can be acquired or released on demand.
    The undisputed power of cloud computing constitutes in its elasticity and granularity: it allows users to ask for hundreds of computers for only 5 minute usage which
    are shipped during several minutes.
    Such services are usually offered over remote network connection and users are billed
    for the portion of the resources they have used.
    Depending on the cloud infrastructure type the payment models can be different~\cite{Laatikainen},
    but the common spendings are associated with data storage, data transfer, and computing timeshare.

    Nowadays the industry increasingly relies on cloud technologies.
    More and more companies start or move their products to cloud environment.
    Unfortunately it comes with additional financial costs imposed by cloud providers.
    In order to make business profitable companies try to reduce amont of money they are supposed to pay to the minimum.
    Despite different solutions, like employing specific cloud cost optimization team, more and more firms decide to benefit from dedicated software, which is supposed to help manage and optimize their cloud usage.
    It is not easy to choose the right tool having that many choices.

    There are some notable solutions of AWS cloud optimization problem -- \textit{AWS Cost Explorer} and \textit{AWS Cost Management}, \textit{Cloudability}, \textit{Apptio} and others widely used nowadays.
    In spite of that fact there is still place for new tools targeting omitted types of clients or wrapping and bundling the greatest features from exisiting ones.

    \section{Aim of the thesis}

    The primary objective of the thesis is to create a tool complementing existing solutions used in cloud cost optimisation.
    We focus on Amazon Web Services platform maintained by Amazon as it is one of the most commonly used.
    In our tool, called \textit{Omigost}, we will try to target small/middle sized companies creating simple and easy to use software with flexible configuration options.

    \section{Structure of the thesis}

    The thesis is structured as follows.
    In Chapter 2 we describe the problem of cloud cost optimization with additional description of selected existing solutions.
    Then, in Chapter 3 we present our solution in detail.
    We mention, among others, whole system architecture, API and configuration.
    Finally, in Chapter 4 we sum up the whole thesis.
    In Appendix A we describe how to introduce our solution in a company.

    \section{Contribution of each author}

    It is important to mention that each author worked to some extent on every part of the thesis.
    However authors contributed mainly to the following parts:

    \begin{itemize}
        \item Michał Ołtarzewski
        \begin{itemize}
            \item Software development process management
            \item Design of backend architecture
            \item Backend part implementation
        \end{itemize}
        \item Michał Balcerzak
        \begin{itemize}
            \item Research of existing solutions
            \item Design of backend architecture
            \item Backend part implementation
        \end{itemize}
        \item Piotr Styczyński
        \begin{itemize}
            \item Design and prototype of visual part of tool
            \item Frontend part implementation
        \end{itemize}
        \item Gor Safaryan
        \begin{itemize}
            \item Research of AWS APIs and SDKs
            \item Design of backend architecture
            \item Backend part implementation
        \end{itemize}
    \end{itemize}

    % \subsection{Cloud resources}
    %
    % Cloud computing has introduced 3 new aspects which are radically different from a traditional
    % computing paradigm in terms of hardware infrastructure.
    % The availability of practically infinite on demand computing resources,
    % which allows users to deploy applications without any kind of resource planning.
    % The elimination of massive initial financial commitment, therefore allowing small companies to increase
    % their hardware usage proportional to their needs.
    % The ability to granularly allocate computing resources on any kind of timeframe
    % (from minutes up to days etc.), as well as the ability to release them as there is no more need.
    %
    % Depending on the service offered by the provider we differentiate 3 main models of cloud computing:
    %
    % \begin{enumerate}
    %     \item Infrastructure as a service (IAAS).
    %     \item Platform as a service (PAAS).
    %     \item Software as a service(SAAS).
    % \end{enumerate}
    %
    % % TODO discuss double dash vs triple dash
    % \subsection{IAAS -- Infrastructure as a service}
    %
    % In such form of service, the provider allocates an instance of virtual machine for the client
    % and ensures that minimal building blocks for the IT infrastructure are present:
    % network, storage, computing capacity, load balancer, VLAN etc.
    % Usually providers of such services run pools of hypervisors such as Xen, VMware, QEMU etc which host and manage
    % those virtual machines. From the user perspective it usually looks like a command line interface
    % through which user has full control over the allocated virtual machine. Some of the well known
    % services are Amazon EC2, Google Compute Engine, Microsoft Azure IAAS.
    %
    % \subsection{PAAS -- Platform as a service}
    %
    % Platform as a service alleviates the need for the developers to manage the operating system and
    % provides programming language specific execution environment as well as the underlying structure
    % (hardware, network, storage etc). This adds another layer of convenience over IAAS making the deployment
    % and the development of applications much more fluent process. While users benefit from the automatic software maintenance,
    % OS security patches etc, they also lose full control over the virtual machine instances. AWS Elastic Beanstalk, Heroku
    % and Google App Engine are some of the most popular PAAS services.
    %
    % \subsection{SAAS -- Software as a service}
    %
    % This form of service takes control of every layer of application and the user has nothing to do with
    % the underlying infrastructure. This is by all means one of the most popular form of cloud computing as
    % it might not require any kind of technical background to be used. Examples of SAAS services are Google
    % Cloud Vision, Google Docs, Microsoft Office 365.

    \chapter{Problem statement}

    \section{Motivation}

    As there are plenty of various billing models for cloud services~\cite{GLaatikainen},
    the effective management of them became a tough problem.
    The ease of resource allocation led to situation when tracking tiniest details of billings is an unaffordable challenge.

    The tooling that exists is targetting wideworld-scale companies that are able to require expensive licenses and hire cost-optimization teams.
    The software as it is in case of Cloudability is too complex for average user and does not provide easy way to incorporate custom business flows into the tool.
    Amazon, as one of the leading cloud providers, offers different tools for exploration of expenses.
    Most commonly used options are either their public APIs~\cite{AWSCostManagement} or specific SDKs supporting lots of languages.
    Unfortuantely previously mentioned tools are rather simple and do not satisfy all the needs of potential clients.

    The common case that is unresolved is the distribution of research and development resources.
    We observed that there exists no tool that would support request for resources of individual worker with regards to custom management propagations as specified by client bussiness model.
    Cloudability~\cite{CloudabilityAlerts} offers simple alerts, but they lack Slack support and beforementioned propagation abilities.

    There exists an obvious gap in the market, that our solution, \textit{Omigost}, will try to cover.
    Having one versatile tool removes the need for using few detached pieces of software.
    The most important aspects of our tool are intuitive interface and different types of user notifications, which highly increases spendings clarity and helps in decisions connected to costs cutting.
    We hope it will allow companies to focus more on providing value to their clients having lower costs in the same time.
    Problem we would like to solve is lack of the tool that has simultaneously the following features:
    \begin{itemize}
        \item Free and Open Source
        \item Easy cloud management without complex knowledge
        \item Intuitive interface for individual workers to request resources
        \item Notifications for cost surpassing and redundant resources
        \item Manage notification well -- only significant alerts
        \item Highly configurable and flexible
        \item Integration with communication via Email and Slack
    \end{itemize}

    \section{Overview of exisiting solutions}

    Businesses that rely upon cloud services often reach a point where resources
    they are using up gradually become less and less manageable.
    As the problem is well known to the cloud market, both Amazon and other third-party companies made attempts
    to fulfill these needs by creating custom software fitting certain roles in optimizing AWS expenses that include:

    \begin{enumerate}
        \item Configuring budget limits and alerting users when they are exceeded
        \item Instance alerting management
        \item Cost analytics
    \end{enumerate}

    Some of the most prominent tools currently available on the market that improve resource management experience for AWS cloud are described in the following sections.
    Every description is supposed to show key features crucial for cost optimization.

    \subsection{AWS Budgets}

    AWS Budgets\cite{AWSDocs} is a part of Amazon Web Services that allows to set limits of a certain types
    that apply to a chosen period of time.
    When a limit is either exceeded, close to be exceeded or is forecasted to exceed the configured threshold
    before the end of that period, the administrator of that account is notified by email.
    Types of resources one can put this kind of a budget on include:

    \begin{enumerate}
        \item Money spent in total or on a certain type of machines.
        \item Utilisation of selected services.
        \item Utilisation or coverage of reserved instances.
    \end{enumerate}

    \subsection{AWS Cost Explorer and AWS Cost Management}

    AWS Cost Explorer\cite{CostExplorer} enables access to all budget data.
    User can define and generate custom reports in a form of a data chart spanning a selected time interval with chosen time granularity of the samples.
    % https://aws.amazon.com/aws-cost-management/aws-cost-explorer/
    AWS Cost Explorer is also a basis for AWS Cost Management, which is basically a set of predefined reports that form an easily accessible dashboard.
    % https://aws.amazon.com/aws-cost-management/

    \subsection{Cloudability}

    Atlassian's Cloudability\cite{Cloudability} delivers a budget system functionality analogous to AWS Budgets
    along with tools for predicting future spendings and presenting the real cost of AWS resources in utilisation.
    In comparison to Amazon's native tools Cloudability also allows management of multiple accounts in the same time.
    It saves effort of having to set up budgets separately in every owned account.

    \subsection{Apptio}

    Apptio\cite{Apptio} provides a set of tools that mainly focuses on analysis of expenses and their forecasts,
    managing them collaboratively, and planning future ones.
    They expose features that make it easier to discover underutilized resource,
    compare spendings with a database of similar benchmarks,
    organize resources into groups to make reports even clearer,
    and offer other useful management utilities.

    \subsection{Stax.io}

    Stax.io's\cite{Stax.io} main focus is to provide insight about cost, wastage, compliance and cloud quality.
    It can analyse how cloud resources are used, measure quality of the way cloud is utilized,
    set up checks for business-compliance of our cloud with several standards
    and give customized advice on what could be optimized to reduce wastage,
    while also allowing for creation of custom views of data.
    Basic tools for budgeting instances, accounts, tags and more, monthly or annually,
    and configuring overspend alerts are also available there.

    \subsection{SnowSoftware}

    SnowSoftware's toolset, alongside fulfilling some of the more specific usecases
    like optimizing usage of software from SAP Software Solutions or optimizing and managing software licenses,
    also has tools that are dedicated to optimizing cloud costs.

    Snow for SaaS attempts to give a holistic view about application usage including, among others,
    how SaaS applications are used on cloud and whether there are zombie virtual machines~\cite{SnowSaaS}.

    Snow Automation Platform suggests approach based on automated and preconfigured provision of resources.
    By pre-giving those resources a decommission date one can avoid issue of zombie instances.
    It is also possible to preconfigure budgets and schedule machine starts and stops to further optimize costs~\cite{SnowBlog}.

    \subsection{Conclusions}

    Using resources available to us, we concluded that most of the competing tools available on the market
    fail to provide both instance budgeting and machine termination automation.
    A number of them also puts most of their emphasis on analysing usage data rather than
    helping with instance management.
    From tools listed above, Snow Automation Platform is the only one that features both budgeting and automating
    machine termination and stopping.
    However, it's approach is not to straightforwardly manage already existing machines,
    but rather to automate their provisioning.

    \section{Our Solution}

    TODO

    % TODO Begin chapter 3 with showing our product and then comparing with the conclusions above


    \chapter{Project development}

    In this section we describe the way we developed the application -- management and organization of work along with supporting tools.

    \section{Version control}

    During development as our version control system we use Git and whole code repository is hosted on GitHub.
    We have chosen this particular service because of a few reasons.

    First of all GitHub \cite{GitHub} is one of the most popular Git hosting service and also has one of the biggest open source communities.
    It is quite important if we consider possibility of future development of our tool even after finishing this thesis.

    The second reason is integration with lots of useful applications.
    It is easy to use it with different services that improve development process like TravisCI \cite{TravisCI} and Slac k\cite{Slack}.

    Last but not least, GitHub has a built in issue tracker.
    It can be highly customised and makes project management easier.
    We created dashboard divided into a few sections, which separate tasks in different stages of development.
    Everything is automated -- depending on actions performed by users like Pull Requests issues are moved between sections.
    In order to help prioritize tasks we introduced issues tagging system.

    \section{Continuous Integration}

    For Continuous Integration we use hosted service -- TravisCI. It allows us to make sure new changes we introduce are compliant with the rest of our codebase.
    There are two reasons we have chosen this service: it is easy to configure using only one YAML file inside repository and it is free for open source projects.
    After every code submission, at the beginning TravisCI performs Smoke Tests trying to build the project and check whether it runs or not.
    Successful build is followed by both backend and frontend tests.

    \section{Communication}

    All of the communication happens via Slack.
    It is a platform allowing team members to communicate without use of email or group SMS.
    Various channels can be created there, with persistent messages history both public and private.
    There is also possibility to reach out to every member directly if needed.
    We use Slack due to the fact it is free and integrates easily with GitHub.

    \section{Organization of work}

    We work in iterations that last around one month.
    At the beginning we define and create new issues.
    During the iteration everyone chooses desirable task, completes it and makes corresponding Pull Request in GitHub.
    At the end we discuss our overall progress and plan next steps.

    In order to let a new feature become part of the repository it has to be accepted.
    It means that Pull Request has to pass Continuous Integration system build and be approved by one of the reviewers.
    Every issue is solved in a separate branch and every PR is merged directly to the master branch, which allows us to group all commits that are part of one feature or issue.

    \section{Contact with the client company -- Sumo Logic}

    Whole project and thesis are developed under Sumo Logic mentorship.
    There is one particular person designated to act as mentor -- Jacek Migdał.
    He makes sure we are provided with every resource we need to
    work on the project without unnecessary breaks.
    Also our mentor helps revolve any ambiguity and gives technical advice.

    Day to day communication with mentor takes place via Slack.
    Additionally approximately twice a month our team meets in Sumo Logic office to review current progress and overcome major difficulties.
    Every new feature is discussed beforehand with the company.

    \chapter{Tool for AWS cost optimization}

    In this chapter we will describe the whole application -- \textit{Omigost}.

    \section{Use cases}

    In this section you can find descriptions of the most common use cases
    grouped by actors. Every use case assumes company in preconfigured in
    AWS Organizations.

    \subsection{Actor -- AWS cost optimization admin}

    Admin is a person designed to have an access to AWS Cost Optimization platform. Most of the interactions happen throught application website.

    First of all admin can configure users inside the application. There are following possible actions using configuration tab:

    \begin{enumerate}
        \item Adding employee
        \item Adding contact to employee (currently only Slack)
        \item Linking account to specific employee
        \item Defining timeframe for machines termination suggestions
    \end{enumerate}
    It is important to mention that that most of the similar actions like editing and deleting are possible.

    Despite configuration admin can have insight into AWS costs looking at dashboards with charts. It is also possible to customize charts to show only subset of costs.

    \subsection{Actor -- Employee}
    Ordinary employee do not have access to the platform itself. Most of the interactions happen throught Slack. There are a few situations when user receives Slack message from an application bot:

    \begin{enumerate}
        \item Surpassing a budget
        \item Budget is forecasted to being surpassed
        \item Machine on AWS works in specific timeframe
    \end{enumerate}

    If any of the budgets linked to the employee's accounts are surpassed employee will receive notification. Additionally if there is a budget defined only on machine tags, someone (defined in the application) will receive notification.

    Message sent by bot is rather simple. It consists of the title, description with budget information and button. Clicking the button redirects to website, which allows employee to request more money from manager. There user can find a place for the request reasoning and submit button.

    \bigskip

    We can imagine example scenario:
    \begin{enumerate}
        \item Budget is being surpassed
        \item Employee receives message from bot on Slack
        \item Employee analyzes the description
        \item Employee decides to request more money and clikcs the button
        \item Employee is redirected to the request website
        \item Employee completes and submits request form
    \end{enumerate}

    \bigskip

    Situation with mechines termination is very similar. If machine linked to the account owned by the employee is running in defined timeframe, notification will be triggered. Message body contains machine description and also termination button. After clicking mentioned button user can instantly terminate corresponding machine.


    \section{Overall architecture}
    Our solution is based on an architecture that is visualized on Figure 4.1

    \begin{figure}[!htb]
      \center{\includegraphics[width=\textwidth] {imgs/arch_diagram.png}}
      \caption{\label{fig:arch-diag} Draft diagram of the tool architecture}
    \end{figure}

    The architecture of our tool can be described as a traditional client-server architecture.
    Actors make requests over the HTTP protocol to the system with a web browser using the frontend client application
    served by the Omigost server or, alternatively, from any kind of a storage service like, for example, Amazon's S3.
    Those requests are then handled by a backend application running on a central server machine.
    During the handling process our application uses both a local instance of a PostgreSQL relational database
    and a connection with selected Amazon Web Services via a SDK library.
    Additionally, end users are notified about budget or instance events related to their cloud activity through Slack.

    \subsection{Role of AWS cloud services}
    Connecting with Amazon's cloud services not only allows the application to access machines and spending data
    that is crucial to be able to fulfill the business requirements,
    but many of them also solve many of the problems we would otherwise have to solve ourselves -- resulting
    in time saved and a codebase that is more concise and easier to maintain.

    Services that are used in the project and their respective roles in it are as follows:
    \begin{itemize}
        \item Identity and Access Management (IAM) -- provision of the main Amazon account ID that is used in other services,
        \item Organizations -- insight into structure of the Organization used by the company, including listing accounts,
        \item Budgets -- service keeping eye on spendings that triggers a SNS notification whenever a budget limit is or is going to be exceeded,
        \item SNS -- alert notifications for backend that let it know about spendings events,
        \item Cost Explorer -- data for graph visualizing spendings in frontend,
        \item EC2/RDS -- information about state of instances running on the account.
    \end{itemize}

    TODO at the end of the project -- check if any other "sublibraries" were used and/or if any other use cases emerged

    \subsection{Backend}

    Backend is the main agent of our application's functionalities.
    It is responsible for fetching or receiving data from either of other connected resources, parsing it and taking appropriate actions.
    Some of its more important tasks include:
    \begin{itemize}
        \item receiving SNS budget alert notifications and notifying appropriate users via Slack,
        \begin{itemize}
            \item providing single use tokens and links for non-admin users to be able to request budget limit increases,
        \end{itemize}
        \item fetching and providing raw data for the frontend so it can be displayed to the user,
        \item receiving configuration or data modification requests and adjusting the application environment components to fulfill them,
        \item checking machine state on preconfigured times and notifying users about possible instance optimizations.
    \end{itemize}

    \subsection{Frontend}
    Frontend is a web application that allows the user to see and alter the state of the application using a graphical user interface (GUI).
    Its main roles are to:
    \begin{itemize}
        \item request and parse raw data from backend into visual representations of it,
        \item provide an interface for the user,
        \item translate the interface clicks to appropriate backend requests and display the results of these requests.
    \end{itemize}

    \section{Technology stack}

    \subsection{Frontend}

    Frontend was written in Typescript \cite{Typescript} using React \cite{React} framework.
    Typescript offers both great flexibility and type safety compared to vanilla Javascript.
    It is the most common alternative to Javascript, supported by large bussiness insitutions like Microsoft.
    We decided that other alternatives i.e Reason (or other functional languages compiled to Javascript) or Dart are useful, but
    still need development and their interoperability is sometimes very limiting factor. All the codebase is lintend using \textit{tslinter}.
    
    Typescript code is transpiled using \textit{tsc} and bundled by \textit{webpack}. We do not use any other build tools like
    \textit{Grunt} or \textit{Gulp} as the backend has its own tooling i.e Gradle. It would just unnecessarly compilicate building process without any gains.
    Frontend build is coordinated by npm plugin for Gradle.

    Frontend codebase is split between universal, reusable components and concrete implementation of user views. That design decision was effect of following the
    general good programming guidelines. It enables the further contributors or users of application to effectively implement customizations or modification 
    in existing code without unnecessarly big work efforts. The views are split between various \textit{modules}. A module is a standalone entity that provides its 
    own button in the sidemenu of applciation. For increased customizability and modularity each of them can be separetly disabled or modified.
    To synchronize the data between separate modules we decided to use \textit{Redux} with its Flux architecture. That allows us to easily persist the application state
    in the local storage of the browser or elsewhere. Redux store is a central storage for information about current modules settings but also serves views routing data,
    states of dialogs and notifications.
    
    We also provided \textit{Jest} test suites as well as documentation generated by \textit{tsdoc}.
    We also decided to utilize \textit{Storybook} to easily design and visually test frontend components, which for users stands for a complement rather than an alternative for Jest and the documentation.
    
    \subsection{Backend}

    Core of the backend is \textit{Java 8}.
    We have selected this programming language because most of us were familiar with it.
    Also this language has huge community and lots of useful libraries and frameworks.

    For creation of the whole backend service we have chosen Spring \cite{Spring} framework.
    It is currently one of the most common choices.
    Spring allows to build web applications imposing usage of design patterns like Model-View-Controller and Dependency Injection.
    It helps to keep code easily testable and well organized.
    Additionally it provides various features out of the box, which speeds up development process.

    Gradle \cite{Gradle} is responsible for build and dependencies management.
    It is easy to use with various IDEs and has flexible configuration.
    The main competitor of Gradle is Maven so obviously we considered both.
    Gradle turned out to be better in terms of performance and usage convenience.

    For data persistence we have chosen PostgreSQL database.
    Relational database enforces having strictly defined data model with validation.
    Performance is not an important aspect for us so there was no need for other types of databases.
    PostgreSQL is a great open source database with strong data integrity and fault-tolerance guaranties.
    Furthermore, AWS lets set up PostgreSQL instance with just few clicks with automatically configured parameters for optimal performance.

    In our application we use the following libraries:
    \begin{enumerate}
        \item Spring Core, Spring Boot, Spring Data, Spring Web,
        \item Project Lombok,
        \item JUnit,
        \item AWS SDKs.
    \end{enumerate}

    \subsection{Deployment}

    The whole tool is bundled up by Docker.
    Following a straightforward instruction everyone is able to create their own Docker
    image of the application.
    Moreover, all of the configuration is present in one file making it easy to adapt it your way.

    A preferable method of deployment is to use AWS Elastic Beanstalk with Docker image.
    In such a way AWS will be responsible for almost everything including capacity provisioning,
    load balancing and auto-scaling.

    \section{Data models}
    Waiting for final notifications implementation
    \section{Views and design}
    
    \subsection{General design}
    
    The main design principles that we used in application design and implementation were:
    \begin{enumerate}
        \item Interface usability is more important than its look
        \item The style should be as simple and minimalistic as possible
        \item Use limited color pallete with few colours that match each other perfectly        
    \end{enumerate}
    
    The Omigost icons were drawn in Inkscape as SVG graphics and we created primitive,
    initial mocks in UXPin \cite{UXPin} tool.
    
    \subsection{Login view}
    
    \includegraphics[width=\textwidth] {imgs/screenshots/screen_login.png}
  
    The login view is designed with geometric simplicity in mind.
    It is showing the Omigost basic color palette basing mostly on red, which we consider
    a default main color that can be associated with the main application logo.
    
    \subsection{Dashboard}
    
    \includegraphics[width=\textwidth] {imgs/screenshots/screen_dashboard.png}

    The very first user experience after the logon screen and the loading screen is the dashboard.
    Here you can add customized widgets to display the AWS costs data.
    The rich and simple interface allows the users to select and add new widgets
    and configure them however the user wants.
    
    \includegraphics[width=\textwidth] {imgs/screenshots/screen_dashboard_add.png}
    
    \subsection{Budgets}
    
    An application user can easily create a new budget,
    change its spending threshold and attach new tags or accounts to it.
    
    \includegraphics[width=\textwidth] {imgs/screenshots/screen_budgets_create.png}
    
    The budget listing presents a visually simple grid with all reported and forecasted
    costs as well as comparison charts to quickly compare spendings.
    
    \includegraphics[width=\textwidth] {imgs/screenshots/screen_budgets_browse.png}
    
    \subsection{Accounts and user management}
    
    The account listing allows user to browse all AWS accounts
    and presents useful details for each of them (i.e ARN, IDs etc.)
    
    \includegraphics[width=\textwidth] {imgs/screenshots/screen_accounts_browse.png}
    
    Similar view exists in the users tab to present all created user accounts in the system.
    
    \includegraphics[width=\textwidth] {imgs/screenshots/screen_users_browse.png}
    
    Creation of the new user is a straigtforward process. The user can easily:
    \begin{enumerate}
        \item Remove or attach new communication channels
          to the created users (i.e. connect Slack or emails account to receive notifications),
        \item Remove or attach new accounts to the user
          (user will be notified when those accounts will be overbudgeted)
        \item Delete users or change their properties
    \end{enumerate}
    
    \includegraphics[width=\textwidth] {imgs/screenshots/screen_users_create.png}
    
    \subsection{Customization}
    
    The views responsible for managing various application settings
    were spread across configuration panels to allow users to easily change
    them without introducing a configuration option mess.
    
    \includegraphics[width=\textwidth] {imgs/screenshots/screen_customize.png}
    
    Here the user can:
    \begin{enumerate}
        \item Change theming options to customize the look and feel of the application,
        \item Change instance settings (i.e. AWS keys, deployment URLs etc.)
        \item Change enabled extensions and integrations
        \item Change current notification settings
    \end{enumerate}
    
    The extension screens allow users to enable or disable
    dashboard extensions (for example hide the budgets view if wanted) or change their settings.
    
    \includegraphics[width=\textwidth] {imgs/screenshots/screen_customize_extensions.png}
    
    The instance settings screen is a straightforward way to manage
    Omigost deployment settings. You can provide your own services URLs for Omigost dependencies
    (like custom AWS Budgets store service) or change AWS and Slack credentials.
    
    \includegraphics[width=\textwidth] {imgs/screenshots/screen_customize_settings.png}
    
    Theming views is useful to change the look and feel of the application
    to make users comfortable with a more friendlier, customized environment.
    
    \includegraphics[width=\textwidth] {imgs/screenshots/screen_customize_theme.png}
    
    \section{Communication integration}
    TODO
    \section{Service termination architecture}
    While developing the application, we kept a certain model of organization of resources in mind, which the enterprise user might have in the AWS cloud. A common way of keeping an organization in AWS cloud is via service called “AWS Organizations” \cite{AWSOrganizations}. The service allows to allocate AWS accounts with the predefined roles and permissions for every member of organization. Such structure also isolates resources from each other and lowers the granular control that the root account has on user allocated resources.

For our application to be able to tweak and monitor every machine created by the organization members, every employee of the organization needs to create a certain predefined role which should have the same name across all accounts. The role should have access to all resources that the user wants to be monitored. This can be done by a simple script every time a new member joins to the organization.

Secondly the account, from which the application will be deployed, needs to create an AWS IAM user \cite{AWSIAM}, which will be used by the application to assume the roles created by other members of the organization. To enable the functionality, the administrator needs to give “AssumeRole” permission to the newly created IAM user.

We recommended to take the following steps to grant the above mentioned permission.



\begin{enumerate}

\item Create an IAM group called “AssumesRolesGroup”.
\item Go to the permissions section of the group.
\item Create a custom group policy
\item Add the following “JSON” to the policy document.

\begin{lstlisting}[language=json,firstnumber=1]
{
  "Version": "2012-10-17",
  "Statement": [
    {
      "Effect": "Allow",
      "Action": [
        "sts:AssumeRole"
      ],
      "Resource": "*"
    }
  ]
}
\end{lstlisting}

\item Add the IAM user to the “AssumesRoleGroup” group.
\end{enumerate}

After taking all those steps, you just need to copy the AWS keys of that IAM user account to the application configuration file and you are good to go.


    \chapter{Summary}
    The purpose of the project was to implement a solution for optimizing cloud costs
    in a typical medium-sized company according to needs reported by Sumologic.
    From two approaches that were suggested by the ordering party,
    we chose one that was more software-focused rather than analytic.

    Development spanned a period of about a half of a year and planning and preparations took us
    a few more months.
    We developed our application in irregular iterations that each of us adjusted
    to their schedule.
    Additionally, we held meetings with our promotor roughly every 2-3 weeks
    as well as with the contractor when we needed to consult development and requirement details.
    The company expressed a huge interest in the implementation of the project
    as well as further development of it.

    (<TODO>)In late stage of the development phase our system was tested
    by a sample group of employees at Sumologic, who provided us additional feedback
    that allowed us to further improve the product.(</TODO>)

    In the end we built a system that focuses on two main features:
    - easily manageable AWS Budgets,
    - automatic machine termination,
    both of which benefit heavily from a Slack integration.

    The product is a web application consisting of Java Spring server and Typescript browser
    frontend.
    The solution is deployed along a Postgres relational database with a Docker container
    on a Beanstalk instance.
    The project is fully open-source, released under MIT license
    and available on Github.


    \begin{thebibliography}{99}
        \addcontentsline{toc}{chapter}{Bibliography}

        \bibitem[Apptio]{Apptio}
        2007-2019 Apptio, Inc.
        \textit{Apptio Fuels Digital Transformation}.
        Apptio toolkit overview.
        \textit{https://www.apptio.com/products}.
        Accessed - 01 April 2019.

        \bibitem[Armbrust]{Armbrust}
        2009 M. Armbrust, A. Fox, R. Griffith, A. D. Joseph, R. H. Katz et al.
        \textit{Above the Clouds: A Berkeley View of Cloud Computing}.
        Technical Report No. UCB/EECS-2009-28.
        \textit{http://www.eecs.berkeley.edu/Pubs/TechRpts/2009/EECS-2009-28.html}.

        \bibitem[AWSCostManagement]{AWSCostManagement}
        2019, Amazon Web Services, Inc. or its affiliates.
        \textit{AWS Cost Management -- Welcome Page}.
        Initial description of the Cost Explorer and AWS Budgets API usage.
        \textit{https://docs.aws.amazon.com/aws-cost-management/latest/APIReference/Welcome.html}.
        Accessed - 01 April 2019.

        \bibitem[AWSDocs]{AWSDocs}
        2019, Amazon Web Services, Inc. or its affiliates.
        \textit{AWS documentation}.
        \textit{https://docs.aws.amazon.com}.
        Accessed - 01 April 2019.

        \bibitem[AWSIAM]{AWSIAM}
        2019, Amazon Web Services, Inc. or its affiliates.
        \textit{AWS Identity and Access Management (IAM) documentation}.
        \textit{https://docs.aws.amazon.com/iam/index.html}.
        Accessed - 01 April 2019.

        \bibitem[AWSOrganizations]{AWSOrganizations}
        2019, Amazon Web Services, Inc. or its affiliates.
        \textit{AWS organizations documentation}.
        \textit{https://docs.aws.amazon.com/organizations/index.html}.
        Accessed - 01 April 2019.

        \bibitem[Cloudability]{Cloudability}
        2019 Cloudability Inc.
        \textit{Cloudability - Platform Overview}.
        Cloudability frontpage, description of its capabilities.
        \textit{https://www.cloudability.com/product/}.
        Accessed - 01 April 2019.

        \bibitem[CloudabilityAlerts]{CloudabilityAlerts}
        % TODO add description
        2015 Leah Weitz.
        \textit{Creating budget alerts by tag with Cloudability}.
        \textit{https://blog.cloudability.com/creating-budget-alerts-by-tag-with-cloudability/}.

        \bibitem[CostExplorer]{CostExplorer}
        2019 Amazon Web Services, Inc. or its affiliates.
        \textit{AWS Cost Explorer}.
        Service Capabilities Description.
        \textit{https://aws.amazon.com/aws-cost-management/aws-cost-explorer/}.
        Accessed - 01 April 2019.

        \bibitem[Docker]{Docker}
        2019 Docker Inc.
        \textit{Docker}.
        Docker tool frontpage.
        \textit{https://www.docker.com/}.
        Accessed - 01 April 2019.

        \bibitem[GitHub]{GitHub}
        2019 GitHub, Inc.
        \textit{GitHub}.
        GitHub tool frontpage.
        \textit{https://github.com/}.
        Accessed - 01 April 2019.

        \bibitem[Laatikainen]{Laatikainen}
        2013 G. Laatikainen, A. Ojala, O. Mazhelis.
        \textit{Cloud Services Pricing Models}.

        \bibitem[Markus]{Markus}
        2011 M. Böhm, S. Leimeisier, C. Riedl, H. Krcmar.
        \textit{Cloud Computing and Computing Evolution}.
        Technische Universität München (TUM), Germany.

        \bibitem[Project Lombok]{Project Lombok}
        2009-2019 The Project Lombok Authors.
        \textit{Project Lombok}.
        Project Lombok library frontpage.
        \textit{https://projectlombok.org/}.
        Accessed - 01 April 2019.

        \bibitem[React]{React}
        2019 Facebook Inc.
        \textit{React}.
        React framework official frontpage.
        \textit{https://reactjs.org/}.
        Accessed - 22 May 2019.
        
        \bibitem[Slack]{Slack}
        2019 Slack Technologies.
        \textit{Slack}.
        Slack tool frontpage.
        \textit{https://slack.com/}.
        Accessed - 01 April 2019.

        \bibitem[SnowBlog]{SnowBlog}
        2017 David Svee.
        \textit{The True Cost of AWS}.
        Overview of cloud cost issues and ways to optimize them.
        \textit{https://www.snowsoftware.com/es/blog/2018/06/16/true-cost-aws}.
        Accessed - 01 April 2019.

        \bibitem[SnowSaaS]{SnowSaaS}
        2018 Snow Software.
        \textit{Snow for SaaS | Snow Software - The Cloud Challenge}.
        Description of Snow for SaaS targets.
        \textit{https://www.snowsoftware.com/int/snow-saas}.
        Accessed - 01 April 2019.

        \bibitem[Spring]{Spring}
        2019 Pivotal Software, Inc.
        \textit{Spring by Pivotal}.
        Spring framework frontpage.
        \textit{https://spring.io/}.
        Accessed - 01 April 2019.

        \bibitem[Stax.io]{Stax.io}
        % TODO add description
        \textit{https://www.getapp.com/it-management-software/a/stax/}.
        \textit{https://www.stax.io/features}.

        \bibitem[TravisCI]{TravisCI}
        2019 Travis CI, GmbH.
        \textit{Travis CI}.
        Travis CI Continuous Integration service frontpage.
        \textit{https://travis-ci.org/}.
        Accessed - 01 April 2019.
        
        \bibitem[Typescript]{Typescript}
        2019 Microsoft
        \textit{Typescript}.
        Typescript language official webpage.
        \textit{https://www.typescriptlang.org/}.
        Accessed - 22 May 2019.

        \bibitem[UXPin]{UXPin}
        2019 UXPin Inc.
        \textit{UXPin}.
        UXPin tool official webpage.
        \textit{https://www.uxpin.com/}.
        Accessed - 28 May 2019.
        
    \end{thebibliography}

\end{document}


%%% Local Variables:
%%% mode: latex
%%% TeX-master: t
%%% coding: latin-2
%%% End:
